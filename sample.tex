\newcommand{\preamblefolder}{./include}       % Preamble folder location
\RequirePackage[l2tabu, orthodox]{nag}        % Checks for obsolete habits and stuff, must be called first.
%\RequirePackage{fixltx2e}                     % Implements LaTeX2 fixes.

% Document class and typography packages
\documentclass[letterpaper,DIV=20]{scrartcl}  % KOMA-Script article class, letterpaper option sets page size and DIV sets margins (remove DIV option if problems arise).
\setkomafont{disposition}{\normalfont}        % Sets KOMA-Script title font.
%\usepackage[margin=1in]{geometry}             % Sets the margins to a full size page, margin=1.5cm is also common but cuts off page numbers.
\usepackage{microtype}                        % Nicer kerning.


% Input encoding and standards packages
\usepackage{isomath}                          % Makes math conform to ISO 80000-2 mathematical style. Includes fixmath and kvoptions packages.
\usepackage[utf8]{inputenc}                   % Sets the input encoding to UTF-8.


% Font packages and commands
\usepackage[sc,osf]{mathpazo}                 % Provides Palatino typeface, options: sc = smallcaps, osf = oldstyle numerals. (rm and math)
\linespread{1.05}                             % Changes the vertical spacing between lines as recommended for Palatino fonts.
\AtBeginDocument{                             % Provides AMS mathbb fonts.
  \DeclareSymbolFont{AMSb}{U}{msb}{m}{n}
  \DeclareSymbolFontAlphabet{\mathbb}{AMSb}
}
\usepackage[scaled]{helvet}                   % Helvetica (ss)
%\usepackage{classico}                        % Classico (ss), broken
%\usepackage{courier}                         % Courier (tt)
%\renewcommand{\ttdefault}{lmtt}              % Latin Modern (tt)
\usepackage[lf]{sourcecodepro}                % Source Code Pro (tt)
\usepackage{mathrsfs}                         % Support for script (\mathscr) fonts.
\usepackage{dtklogos}                         % Provides logos like \LaTex.
\normalfont                                   % necessary to make LaTeX aware of our choice of the base font
\usepackage[T1]{fontenc}                      % Tells LaTeX to use T1 encoded version of fonts when available


% AMS math packages
\usepackage{amsmath}                          % AMS math.
\usepackage{amsthm}                           % AMS theorem.
\usepackage{amssymb}                          % AMS mathbb and other stuff.
\usepackage{mathtools}                        % Extension to amsmath, provides \DeclarePairedDelimiterX.


% Functionality providing packages
%\usepackage{scrextend}                       % KOMA-Script package, provides addmargin environment, not necessary if using KOMA-Script article class.
\usepackage{color}                            % Provides text-mode only text coloring.
\usepackage{cancel}                           % Provides \cancel for crossing out other symbols.
\usepackage{esvect}                           % Lets you typeset vector arrows with \vv{AB}.
\usepackage[retainorgcmds]{IEEEtrantools}     % IEEEeqnarray environment.
\usepackage{wrapfig}                          % Provides environment to let text wrap around figures.
\usepackage{verbatim}                         % Provides a block comment environment.
\usepackage{enumitem}                         % Fancy enumerate environment.
\usepackage{array}                            % Fancy array environment.
\usepackage[pdfusetitle]{hyperref}            % Provides support for hyperlinks.
\usepackage{amsrefs}                          % Provides an environment for handling references and citations, needs to be loaded after hyperref.
\usepackage{bussproofs}                       % Provides environments for writing Gentzen style proofs.
\usepackage{float}                            % Provides float positioning for images (via \begin{figure}[H])
           % Main packages
\usepackage{tikz} %tikz graphics
\usetikzlibrary{cd}
%\usetikzlibrary{arrows,automata,shapes}
           % Tikz packages
%special functions
%Set Theory
\DeclarePairedDelimiterX{\card}[1]{\lvert}{\rvert}{#1} %cardinality
\DeclareMathOperator{\pow}{\mathcal{P}} %power set
%Graph Theory
\DeclareMathOperator{\gver}{V}
\DeclareMathOperator{\gedg}{E}
\DeclareMathOperator{\gdis}{d}
\DeclareMathOperator{\gdeg}{deg}
\DeclareMathOperator{\gclo}{C}
%Topology
\DeclareMathOperator{\tn}{N} %Neighborhood
\DeclareMathOperator{\tbd}{bd} %Boundary points
\DeclareMathOperator{\tint}{int} %Interior points
\DeclareMathOperator{\tacc}{acc} %Accumulation points
\DeclareMathOperator{\tiso}{iso} %Isolated points
\DeclareMathOperator{\tcl}{cl} %Closure
\DeclareMathOperator{\sgn}{sgn}
\DeclareMathOperator{\dis}{dis} %Discrete topology
\DeclareMathOperator{\indis}{indis} %Indiscrete topology
\DeclareMathOperator{\dist}{d} %distance (metric)
\DeclareMathOperator{\der}{Der} %derived set
%Analysis
\newcommand{\ud}{\,\mathrm{d}}
\DeclarePairedDelimiterX{\norm}[1]{\lVert}{\rVert}{#1}
\renewcommand*{\Vec}[1]{\vv{\mathbf{#1}}}%
%Number Theory
\DeclareMathOperator{\lcm}{lcm}
%Computability
\newcommand{\monus}{\stackrel{{}^{\scriptstyle .}}{\smash{-}}} % monus
\DeclareMathOperator{\lt}{lt}
\DeclareMathOperator{\halted}{halted}


%theorems, definitions, lemmas
\newtheorem*{thm}{Theorem}
\newtheorem*{defn}{Definition}
\newtheorem*{cor}{Corollary}
\newtheorem*{lem}{Lemma}
%\newtheorem*{ore}{Ore's Theorem}


%special symbols/commands
\newcommand{\contradiction}{{\hbox{%
    \setbox0=\hbox{$\mkern-3mu\times\mkern-3mu$}%
    \setbox1=\hbox to0pt{\hss$\times$\hss}%
    \copy0\raisebox{0.5\wd0}{\copy1}\raisebox{-0.5\wd0}{\box1}\box0
}}} %contradiction hash
%\newcommand{\scontradiction}{\textxswup} %swords
\newcommand*\Heq{\ensuremath{\overset{\kern2pt l'H}{=}}}
\newcommand{\ts}{\textsuperscript} % text version of ^
\newcommand*{\mathcolor}{} %mathcolor
\def\mathcolor#1#{\mathcoloraux{#1}} %mathcolor
\newcommand*{\mathcoloraux}[3]{%
  \protect\leavevmode
  \begingroup
    \color#1{#2}#3%
  \endgroup
} %mathcolor
\newcommand{\aspace}{\vspace{2.5\baselineskip}} % Used for spaces between a question and its answer.
\newcommand{\qspace}{\vspace{1.5\baselineskip}} % Used for spaces between different questions.
%\newcommand{\saspace}{\vspace{1.5\baselineskip}}
%\newcommand{\sqspace}{\vspace{0.5\baselineskip}}


%Alternate QED symbols.
%\renewcommand{\qedsymbol}{$\contradiction$}
%\renewcommand{\qedsymbol}{\textxswup}
%\renewcommand{\qedsymbol}{$\square$}
       % Functions and commands

\usepackage[enable]{easy-todo}                % enable = todo messages are visible, disable = todo messages are invisible.

%--------------title stuff

%Here are three different ways of setting the title and author on a document.

%\title{MATH 999 -- Assignment 9}
%\author{
%  Last-Name, First-Name\\
%  \texttt{email@domain.com}
%}

\title{MATH 999 -- Assignment 9}
\author{
  Last-Name, First-Name\\
  \texttt{email@domain.com}\\
 \and
 Last-Name, First-Name\\
 \texttt{email@domin.com}
}

%\title{MATH 999 -- Assignment 9}
%\author{
%  Last-Name, First-Name\\
%  \texttt{email@domain.com}\\
%   {\small Collaborators:
%   Last-Name, First-Name;
%   Last-Name, First-Name;
%   Last-Name, First-Name}
%}
%--------------\title stuff

\begin{document}
  %--------------header stuff
  \maketitle
  %--------------\header stuff

    \begin{enumerate}
      %Question 1
      \item[\textbf{2.IV}.] % Fancy lables thanks to enumitem environment.
        Provide some examples of proofs.

        \aspace

        Here is a proof with a contradiction symbol.
        %We will do this by calling \renewcommand on the QED symbol (the default is $\square$), this lets me temporarily switch over to a contradiction symbol (defined in my preamble). I will set it back to $\square$ after the proof.
        \begin{thm}
          If $Q$ is both true and false then P is true.
        \end{thm}
        \renewcommand{\qedsymbol}{$\contradiction$}
        \begin{proof}
          Suppose $P$ is false.

          However, $Q$ is both true and false.

          Therefore we have a contradiction and thus $P$ must be true.
        \end{proof}
        \renewcommand{\qedsymbol}{$\square$}

        Notice that the theorem environment (the AMS standard approach to writing theorems) manages its own whitespace and emphasizes the text in italics. I'm not sure why but this doesn't seem to play nice with some of my other packages (note the unusual spacing throughout this document). Because of this I often actually just write \textbf{Theorem.} instead of using an environment though this is considered very bad practice since it loses theorem numbering and other features (the \emph{correct} way to handle this is to define my own theorem environment).

      \qspace

      %Question 2
      \item[\textbf{2.XI}.]
        Write a large proof with several lemmas and a corollary.

        \aspace

        Typically a person writes lemmas before getting to the proof itself in order to avoid confusion about what is being proven. However by using indentation it is possible to write a lemma inside the proof body (as a sort of nested subproof) without creating confusion. As an added bonus it eliminates any confusion inflicted on the reader when they are unable to figure out the reason for needing the lemma until getting to the proof itself. This style is very unusual, so if it looks strange to you then perhaps it's better not to use it.

        When using indenting, the rule of thumb I use is that any lemma that stands on its own and any lemma that many proofs rely on should be written on its own outside the proof. Otherwise, any lemma that solely exists to support a specific proof should be written within the proof body (unless the lemma is huge or something). I find this akin to writing a program where some functions are defined inside the main functions and others are defined elsewhere. The analogy could be pushed further if one acknowledges the Curry-Howard correspondence.

        First we prove lemma 1.

        \begin{lem}
          Lemma proof statement.
        \end{lem}
        \begin{proof}
          Proof Body. \todo{Write out this proof}
        \end{proof}

        Next we prove lemma 2.

        \begin{lem}
          Lemma proof statement.
        \end{lem}
        \begin{proof}
          Proof Body. \todo{Write out this proof}
        \end{proof}

        Now we can move onto our main result.

        \begin{thm}
          Main result proof statement.
        \end{thm}
        \begin{proof}
          We need to show that a number of things are satisfied.

          We can do this with lemmas.

          First we prove a nested sublemma 1.
          \begin{addmargin}{1cm}
            \renewcommand{\qedsymbol}{$\contradiction$}
            \begin{lem}
              Sublemma proof statement.
            \end{lem}
            \begin{proof}
              Proof body. \todo{Write out this proof}
            \end{proof}
            \renewcommand{\qedsymbol}{$\square$}
          \end{addmargin}

          Next we prove a nested sublemma 2.
          \begin{addmargin}{1cm}
            \begin{lem}
              Sublemma proof statement.
            \end{lem}
            \begin{proof}
              Proof body. \todo{Write out this proof}
            \end{proof}
          \end{addmargin}

          Finally, by taking all of our lemmas together with the reference ~\cite{Steen} we can see the result is proven.
        \end{proof}

        Now we can include a corollary.
        \begin{cor}
          Corollary proof statement.
        \end{cor}
        \begin{proof}
          Corollary proof body.
        \end{proof}
        
      \qspace

      %Question 3
      \item[\textbf{2.XII}.]
        Format a proof with cases.

        \aspace

        Cases are also very difficult to handle without using indentation. They are the reason I began using the \textit{addmargin} environment, though a better approach would be to define an environment for handling cases.

        \begin{lem}
          Proof statement.
        \end{lem}
        \begin{proof}
          We can break the problem into cases.

          \textbf{Case 1:} ($x<0$)
          \begin{addmargin}{1cm}
            Suppose $x=0$.

            Do things.

            Thus $P$ is true.
          \end{addmargin}

          \textbf{Case 2:} ($x=0$)
          \begin{addmargin}{1cm}
            Suppose $x=0$.

            Do things.

            Thus $P$ is true.
          \end{addmargin}

          \textbf{Case 3:} ($x>0$)
          \begin{addmargin}{1cm}
            Suppose $x>0$.

            Do things.

            Thus $P$ is true.
          \end{addmargin}

          Since $P$ is true in each case then $P$ must always be true.
        \end{proof}

      \qspace

      %Question 4
      \item[\textbf{2.XII}.]
        Format a proof by induction.

        \aspace

        Proofs by induction can be difficult and tedious to typeset and as such often just get handwaved away. I haven't quite settled on a standard way of writing them but this style seems to work.

        \begin{thm}
           Proof Statement.
        \end{thm}
        \begin{proof} We proceed by induction.

        \textit{Basis:} Suppose $n=0$.

        Do things.

        Therefore the basis is proven.

        \textit{Inductive step:} Let $k\in\mathbb{N}$ be an arbitrary integer. Suppose the following statement
        \begin{addmargin}{1cm}
          \textit{Inductive hypothesis:} $f(k)\leq f(k+1)$.
        \end{addmargin}
        in order to prove the following statement
        \begin{addmargin}{1cm}
          \textit{Inductive claim:} $f((k+1))\leq f((k+1)+1)$.
        \end{addmargin}

        Do things.

        This proves the inductive claim and completes the inductive step.

        The basis together with the inductive step prove our original claim by induction thus completing the proof.
        \end{proof}

      \qspace

      %Question 5
      \item[\textbf{3.II}.]
        Format an equation array.

        \aspace

        Here are two different ways of writing an equation array (there exist many more).

        The \textit{IEEEeqnarray} version looks a little better and has slightly better functionality when it comes to line referencing and labels but it is more cumbersome to write than the \textit{align} version.

        \textbf{Note:} I'm using the \textit{enumitem} package for the \textit{enumerate} environment. It lets me manually define the numbering format and even give a crazy explicit label like this.

        \begin{enumerate}[label = --\alph*--$>$]

        \item
          This is written using the align environment.

          \begin{align*}
            4xyzw
            &= 2\cdot2tu \\
            &\le 2\cdot(t^2+u^2)                    \tag{a remark in parentheses} \\
            &= 2\cdot((xy)^2+(zw)^2) \\
            &= 2\cdot(x^2y^2+z^2w^2)                \tag*{a remark without parentheses} \\
            &= 2x^2y^2+2z^2w^2 \\
            &\le ((x^2)^2+(y^2)^2)+((z^2)^2)+(w^2)^2) \\
            &= x^4+y^4+z^4+w^4
          \end{align*}

        \item
          This is the same math written using the IEEEeqnarray environment.

          \begin{IEEEeqnarray*}{rCl"s}
            4xyzw
            &=& 2\cdot2tu             & \\
            &\le& 2\cdot(t^2+u^2)     & (a remark in parentheses) \\
            &=& 2\cdot((xy)^2+(zw)^2) & \\
            &=& 2\cdot(x^2y^2+z^2w^2) & a remark without parentheses \\
            &=& 2x^2y^2+2z^2w^2       & \\
            &\le& ((x^2)^2+(y^2)^2)+((z^2)^2)+(w^2)^2) & \\
            &=& x^4+y^4+z^4+w^4
          \end{IEEEeqnarray*}

        \end{enumerate}

      \qspace

      %Question 6
      \item[\textbf{3.II Bonus}.]
        Write some equations with cases.

        \aspace

        Here is function with a case in it. It's possible to define these using cases but using the IEEEeqnarraybox is more flexible, looks pretty nice, and is really easy to copy and paste.
          \begin{equation*}
            F(x_2) = \left\{ \,
            \begin{IEEEeqnarraybox}[][c]{l?s}
              \IEEEstrut
              \#\mathscr{P}_{x_2} & if $x_2\in\text{Total}$, \\
              x_2 & if $x_2\notin\text{Total}$.
              \IEEEstrut
            \end{IEEEeqnarraybox}
            \right.
          \end{equation*}

          Here is another function so you can see which parts change depending on the function.
          \begin{equation*}
            F_n = \left\{ \,
            \begin{IEEEeqnarraybox}[][c]{l?s}
              \IEEEstrut
              0 & if $n=0$, \\
              1 & if $n=1$, \\
              F_{n-2} + F_{n-1} & if $n\geq 2$.
              \IEEEstrut
            \end{IEEEeqnarraybox}
            \right.
          \end{equation*}

      \qspace

      %Question 7
      \item[\textbf{4.I}.]
        Draw some commutative diagrams.

        \aspace

        Here are some examples of commutative diagrams. They're made using \textit{tikz} with the tikzlibrary \textit{cd}.

        \begin{enumerate}
          \item[\textbf{Simple:}]
            First a simple diagram is shown.

            \begin{equation*}
              \begin{tikzcd}
                A \arrow[r] \arrow[rd] & B \arrow[d] \arrow[rd] \\
                            & C \arrow[r] & D
              \end{tikzcd}
            \end{equation*}

          \item[\textbf{Complex:}]
            Next a more complex diagram is shown.

            \begin{equation*}
              \begin{tikzcd}[row sep=scriptsize, column sep=scriptsize]
                & f^* E_V \arrow[dl] \arrow[rr, red] \arrow[dd, Rightarrow] & & E_V \arrow[dl] \arrow[dd] \\
                f^* E \arrow[rr, crossing over, blue] \arrow[dd] & & E \\
                & U \arrow[dl] \arrow[rr, bend left=20] \arrow[rr, bend right=20] & & V \arrow[dl] \\
                M \arrow[rr, "\varphi"] & & N \arrow[from=uu, crossing over]\\
              \end{tikzcd}
            \end{equation*}

        \end{enumerate}

    \end{enumerate}

  %--------------footer stuff
  \listoftodos

% These references are managed by the AMSrefs package. Here are a bunch of examples I quickly grabbed.
% A quick and easy way of getting citation information is to look up the book in www.google.scholar.com and click on the link that says cite.
% To cite a reference in text is easy, simply write ~\cite{Steen} or ~\cite{Dasgupta} or whatever. Look above for an example.
  \begin{bibdiv}
    \begin{biblist}
      \bib{Steen}{book}{
        title={Counterexamples in Topology},
        author={Steen, Lynn Arthur},
        author={Seebach, J. Arthur, Jr.},
        volume={18},
        year={1978},
        publisher={Springer}
      }
      \bib{Dasgupta}{book}{
        title = {Set Theory: With an Introduction to Real Point Sets},
        author = {Dasgupta, Abhijit},
        year = {2014},
        publisher = {Springer Science \& Business Media}
      }
      \bib{Landau}{book}{
        title = {Foundations of Analysis},
        volume={79},
        author = {Landau, Edmund},
        date = {1966},
        publisher = {American Mathematical Soc.},
        address = {Providence}
      }
      \bib{Suppes}{book}{
        title = {Axiomatic Set Theory},
        author = {Suppes, Patrick},
        date = {1972},
        publisher = {Dover Publications, Inc.},
        address = {New York}
      }
      \bib{Davis}{book}{
        title = {Computability, Complexity, and Languages: Fundamentals of Theoretical Computer Science},
        author = {Davis, Martin},
        author = {Sigal, Ron},
        author = {Weyuker, Elaine J},
        year = {1994},
        publisher = {Academic Press}
      }
      \bib{Munkres}{book}{
        title = {Topology: A First Course},
        author = {Munkres, James R},
        year = {1975},
        volume = {23},
        publisher = {Prentice-Hall Englewood Cliffs, NJ}
      }
      \bib{Stack}{misc}{
        title = {Show that the infinite intersection of nested non-empty closed subsets of a compact space is not empty},
        author = {MSKfdaswplwq (http://math.stackexchange.com/users/42887/mskfdaswplwq)},
        note = {URL: \url{http://math.stackexchange.com/q/337397} (version: 2013-03-21)},
        organization = {Mathematics Stack Exchange}
      }
      \bib{Lay}{book}{
        title = {Analysis: with an introduction to proof},
        author = {Lay, Steven R},
        year = {2014},
        publisher = {Pearson Education}
      }
     \bib{Rudin}{book}{
        title = {Principles of Mathematical Analysis},
        author = {Rudin, Walter},
        volume = {3},
        year = {1964},
        publisher = {McGraw-Hill New York}
      }
      \bib{Keisler}{book}{
        title = {Foundations of infinitesimal calculus},
        volume={20},
        author = {Keisler, H. Jerome},
        date = {1976},
        publisher = {Prindle, Weber \& Schmidt},
        address = {Boston}
      }
      \bib{O'Connor}{article}{
        title = {An Introduction to Smooth Infinitesimal Analysis},
        author = {Michael O'Connor},
        date = {2008},
        journal = {ArXiv},
        eprint = {http://arxiv.org/abs/0805.3307}
      }
      \bib{Bauer}{article}{
        title = {Intuitionistic Mathematics and Realizability in the Physical World},
        author = {Bauer, Andrej},
        book = {
          title = {A computable universe: understanding and exploring nature as computation},
          author = {Penrose, Roger and Zenil, Hector},
          date = {2013},
          publisher = {World Scientific}
        },
        eprint = {http://math.andrej.com/wp-content/uploads/2014/03/real-world-realizability.pdf},
        pages = {143--158}
      }
    \end{biblist}
  \end{bibdiv}
  %--------------\footer stuff

\end{document}
